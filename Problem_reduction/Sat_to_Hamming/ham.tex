\documentclass{article}

\usepackage[utf8]{inputenc}
\usepackage[T1]{fontenc}
\usepackage[french]{babel}
\usepackage{ntheorem}
\usepackage{amsmath}
\usepackage{amssymb}
\usepackage{multicol}
\usepackage[ landscape, a4paper, hmargin={2cm, 2cm}, vmargin={0.1cm, 2cm}]{geometry}

\usepackage{tikz}
\usetikzlibrary{calc}

\title{Exemple de réduction de SAT vers Détection de Graphe hamiltonien}
\date{}
\author{Valeran MAYTIE}

\begin{document}
  \maketitle
  \begin{center}
  \begin{tikzpicture}{arrowhead=10cm}
    \tikzset{};

    \foreach \r in {0,...,2} {
      \foreach \s in {2, 3, 5, 6, 8, 9, 11, 12} {
        \node [circle, fill=yellow, minimum size =0.5cm] (L\r-\s) at (\s*1.4, \r*2.4) { };
      };
      \foreach \s in {1, 4, 7, 10, 13} {
        \node [circle, fill=green, minimum size =0.5cm] (L\r-\s) at (\s*1.4, \r*2.4) { };
      };
      \node [circle, fill=black, minimum size =0.5cm] (L\r-0) at (0, \r*2.4) { };
      \node [circle, fill=black, minimum size =0.5cm] (L\r-14) at (19.6, \r*2.4) { };
    }
    \foreach \j\r in {0/2,1/1,2/0} {
      \pgfmathtruncatemacro{\rr}{\r + 1};
      \node at (-1.5, \j*2.4) {$V_\rr$};
    }
    

    \node [draw, circle, minimum size=0.5cm] (K1) at (3.5, -2) {$K_1$};
    \node [draw, circle, minimum size=0.5cm] (K2) at (7.7, -2) {$K_2$};
    \node [draw, circle, minimum size=0.5cm] (K3) at (11.9, -2) {$K_3$};
    \node [draw, circle, minimum size=0.5cm] (K4) at (16.1, -2) {$K_4$};

    \foreach \s in {0,...,13} {
      \pgfmathtruncatemacro{\ss}{\s + 1};
      \draw [<-] (L0-\s.335) -- (L0-\ss.205);
      \draw [<-] (L1-\s.335) -- (L1-\ss.205);
      \draw [->] (L2-\s.25) -- (L2-\ss.155);
    }
    \foreach \s in {0,1,3,4,6,7,9,10,12,13} {
      \pgfmathtruncatemacro{\ss}{\s + 1};
      \draw [->, color=red] (L0-\s.25) -- (L0-\ss.155);
      \draw [->, color=red] (L1-\s.25) -- (L1-\ss.155);
      \draw [<-, color=red] (L2-\s.335) -- (L2-\ss.205);
    }
    \foreach \r\s in {0/5,0/11, 1/2,1/5,1/8} {
      \pgfmathtruncatemacro{\ss}{\s + 1};
      \draw [->, color=red] (L\r-\s.25) -- (L\r-\ss.155);
    }
    \draw [<-, color=red] (L2-2.335)   -- (L2-3.205);
    \draw [<-, color=red] (L2-8.335)   -- (L2-9.205);
    \draw [<-, color=red] (L2-11.335)  -- (L2-12.205);
    
    \draw [->, dashed, color=red] (L2-0) to (L1-0);
    \draw [->, dashed] (L2-14) to (L1-14);
    \draw [->, dashed] (L2-0.south east) to (L1-14.north west);
    \draw [->, dashed] (L2-14.south west) to (L1-0.north east);

    \draw [->, dashed] (L1-0) to (L0-0);
    \draw [->, dashed] (L1-14) to (L0-14);
    \draw [->, dashed] (L1-0.south east) to (L0-14.north west);
    \draw [->, dashed, color=red] (L1-14.south west) to (L0-0.north east);

    \draw [->, dashed, bend left] (L0-0) to (L2-0);
    \draw [->, dashed, bend right, color=red] (L0-14) to (L2-14);

    \coordinate[shift={(-1,1)}] (n) at (L2-0);
    \coordinate[shift={(8,1)}] (n1) at (L2-0);
    \draw[dashed, ->, rounded corners=10pt] 
      (L0-0) -| (n) -- (n1) -- (L2-14.north west);

    \coordinate[shift={(1,1)}] (n) at (L2-14);
    \coordinate[shift={(-8,1)}] (n1) at (L2-14);
    \draw[dashed, ->, rounded corners=10pt] 
      (L0-14) -| (n) -- (n1) -- (L2-0.north east);

    \draw[color=red, ->] 
      (L0-2) edge (K1)
      (K1) edge (L0-3);
    \coordinate[shift={(-1.4,1)}] (n1) at (K1);
    \coordinate[shift={(1.4,1)}] (n2) at (K1);
    \draw[color=blue, ->, rounded corners=15pt] 
      (L1-2) -- (n1) -- (K1);
    \draw[color=blue, ->, rounded corners=15pt]
      (K1) -- (n2) -- (L1-3);
    \coordinate[shift={(-1.8,1)}] (n1) at (K1);
    \coordinate[shift={(1.8,1)}] (n2) at (K1);
    \draw[color=blue, ->, rounded corners=15pt] 
      (L2-2) -- (n1) -- (K1);
    \draw[color=blue, ->, rounded corners=15pt]
      (K1) -- (n2) -- (L2-3);
  
    \coordinate[shift={(-1.4,1)}] (n1) at (K2);
    \coordinate[shift={(1.4,1)}] (n2) at (K2);
    \draw[color=blue, ->, rounded corners=15pt] 
      (K2) -- (n1) -- (L1-5);
    \draw[color=blue, ->, rounded corners=15pt]
      (L1-6) -- (n2) -- (K2);
    \coordinate[shift={(-1.8,1)}] (n1) at (K2);
    \coordinate[shift={(1.8,1)}] (n2) at (K2);
    \draw[color=red, ->, rounded corners=15pt] 
      (K2) -- (n1) -- (L2-5);
    \draw[color=red, ->, rounded corners=15pt]
      (L2-6) -- (n2) -- (K2);

    \draw[color=red, ->] 
      (L0-8) edge (K3)
      (K3) edge (L0-9);
    \coordinate[shift={(-1.4,1)}] (n1) at (K3);
    \coordinate[shift={(1.4,1)}] (n2) at (K3);
    \draw[color=blue, ->, rounded corners=15pt] 
      (L1-9) -- (n2) -- (K3);
    \draw[color=blue, ->, rounded corners=15pt]
      (K3) -- (n1) -- (L1-8);

    \draw[color=blue, ->] 
      (L0-12) edge (K4)
      (K4) edge (L0-11);
    \coordinate[shift={(-1.4,1)}] (n1) at (K4);
    \coordinate[shift={(1.4,1)}] (n2) at (K4);
    \draw[color=red, ->, rounded corners=15pt] 
      (L1-11) -- (n1) -- (K4);
    \draw[color=red, ->, rounded corners=15pt]
      (K4) -- (n2) -- (L1-12);
  \end{tikzpicture}
  \end{center}

    \begin{multicols}{5}

      \underline{Formule :} 
      \begin{align*} 
        & V_1 \vee V_2 \vee V_3          & (K_1) \\
        &\wedge \neg V_1 \vee \neg V_2  & (K_2) \\
        &\wedge \neg V_2 \vee V_3       & (K_3) \\
        &\wedge V_2 \vee \neg V_3       & (K_4) \\ \\ \\ \\ \\ \\
      \end{align*}

      Les variables de la formule sont symbolisées par les lignes, 
      au nombre de trois dans cet exemple ($V_1$, $V_2$, $V_3$),
      tandis que les clauses sont représentées par les n\oe uds $K_1, K_2, K_3$ et $K_4$.

      L'objectif de la réduction est de transformer la formule logique 
      en un graphe tel que si celui-ci est reconnu comme hamiltonien, 
      alors la formule est considérée comme satisfaisable (on veut aussi la réciproque).

      \columnbreak

      On a construit le graphe qui représente la formule de gauche.
      Le cycle hamiltonien est dessiné en rouge.
      
      En examinant le sens de déplacement sur la ligne correspondante 
      à une variable, il est possible de récupérer sa valeur. 
      Si la direction est de droite à gauche, la variable est vraie ; 
      dans le cas contraire, elle est fausse.

      \columnbreak

      On a donc :

      \begin{tabular}{c c c c}
        & $V_1$ & $V_2$ & $V_3$ \\
        \hline
        $\mathcal{M}$ & $\mathbf{F}$ & $\mathbf{T}$ & $\mathbf{T}$
      \end{tabular}

      Il est relativement aisé de vérifier 
      la véracité de la formule associée au modèle $\mathcal{M}$.

      Il est observé que les nœuds $K_i$ sont connectés aux variables 
      qui sont présentes dans la clause. 
      Selon la direction de la variable, il peut y avoir 
      deux types de connexions possibles : si la variable apparaît de droite à gauche, 
      sa négation sera présente dans la formule, tandis que si elle apparaît de gauche à droite, 
      elle sera présente sans négation. 
      On peut noter une corrélation entre la direction empruntée par la ligne représentant 
      la variable dans le cycle et la direction de la clause.

      Les nœuds verts sont utiles pour la réciproque, car ils permettent 
      d'éviter la situation où le graphe est hamiltonien, mais la formule n'est pas satisfaisable.
    \end{multicols}
\end{document}
