\documentclass{article}

\usepackage[utf8]{inputenc}
\usepackage[T1]{fontenc}
\usepackage[french]{babel}
\usepackage{ntheorem}
\usepackage{amsmath}
\usepackage{amssymb}
\usepackage{multicol}
\usepackage[landscape, a4paper, hmargin={2cm, 2cm}, vmargin={0.1cm, 2cm}]{geometry}

\usepackage{tikz}
\usetikzlibrary{calc}
\usetikzlibrary{arrows}
\usetikzlibrary{shapes.geometric}

\usepackage{lipsum}% just to generate some text


\title{Exemple de réduction de SAT vers 3 Coloriage}
\date{}
\author{Valeran MAYTIE}

\begin{document}
  \maketitle

  \begin{center}
    \begin{tikzpicture}
      \node[ellipse, fill=green] (C) at (15,0) 
        {$\neg V_1 \vee \neg V_2 \neg V_3 \vee \neg V_4 \vee V_6$};

      \node [circle, minimum size =0.5cm, fill=red] (G4-1) at (14, -1) { };
      \node [circle, minimum size =0.5cm, fill=black] (G4-2) at (16, -1) { };
      \draw (C) -- (G4-1) -- (G4-2) -- (C);

      \foreach \s\cv\cf in 
        {1/green/red, 2/red/green,3/red/green,4/green/red,5/red/green,6/red/green} {
          \node [circle, minimum size =0.5cm, fill=\cv] (V\s-V) at (\s*4, -6) { };
        \node [circle, minimum size =0.5cm, fill=\cf] (V\s-F) at (\s*4+2, -6) { };
        \node [] at (\s*4, -6.5) {$V_\s$};
        \node [] at (\s*4+2, -6.5) {$\neg V_\s$};
        \draw (V\s-F) -- (V\s-V);
      };

      \coordinate[shift={(0,1)}] (n) at (V1-F);
      \node [circle, minimum size =0.5cm, fill=black] (G1-1) at (n) { };
      \coordinate[shift={(0,1)}] (n) at (V2-F);
      \node [circle, minimum size =0.5cm, fill=red] (G1-2) at (n) { };
      \coordinate[shift={(2,1)}] (n) at (G1-1);
      \node [circle, minimum size =0.5cm, fill=green] (G1-3) at (n) { };
      \draw (G1-1) -- (G1-2) -- (G1-3) -- (G1-1) -- (V1-F);
      \draw (G1-2) -- (V2-F);

      \coordinate[shift={(0,1)}] (n) at (G1-3);
      \node [circle, minimum size =0.5cm, fill=black] (G2-1) at (n) { };
      \coordinate[shift={(0,3)}] (n) at (V3-F);
      \node [circle, minimum size =0.5cm, fill=red] (G2-2) at (n) { };
      \coordinate[shift={(3,1)}] (n) at (G2-1);
      \node [circle, minimum size =0.5cm, fill=green] (G2-3) at (n) { };
      \draw (G2-1) -- (G2-2) -- (G2-3) -- (G2-1) -- (G1-3);
      \draw (G2-2) -- (V3-F);

      \coordinate[shift={(0,1)}] (n) at (V4-F);
      \node [circle, minimum size =0.5cm, fill=black] (G3-1) at (n) { };
      \coordinate[shift={(0,1)}] (n) at (V6-V);
      \node [circle, minimum size =0.5cm, fill=green] (G3-2) at (n) { };
      \coordinate[shift={(3,1)}] (n) at (G3-1);
      \node [circle, minimum size =0.5cm, fill=red] (G3-3) at (n) { };
      \draw (G3-1) -- (G3-2) -- (G3-3) -- (G3-1) -- (V4-F);
      \draw (G3-2) -- (V6-V);

      \draw (G2-3) -- (G4-1);
      \draw (G3-3) -- (G4-2);

      \draw(4,-1.9)node[circle, minimum size=0.5cm, fill=green](G) {}
            --++(0:2)node[circle, minimum size=0.5cm, fill=black](B) {}
            --++(120:2)node[circle, minimum size=0.5cm, fill=red](R) {}
            --cycle;

      \draw (R) -- (C.west);
      \draw (B) -- (C.west);
      \draw (B) -- (G1-3);
      \draw (B) -- (G2-3);
      \draw (B) to[bend right=10] (G3-3);

    \end{tikzpicture}
  \end{center}

  \begin{multicols}{5}

    Ce petit graphe est utilisé pour imposer des contraintes de couleurs. :

    \noindent\begin{minipage}{\linewidth}
    \centering
    \begin{tikzpicture}
      \node[circle, fill=green] (G1) at (0, 0) {};
      \node[circle, fill=black] (G2) at (0:1) {};
      \node[circle, fill=red] (G3) at (60:1) {};
      \draw (G1) -- (G2) -- (G3) -- (G1);
    \end{tikzpicture}
    \end{minipage}

    Pour éviter qu'un n\oe ud ne soit colorié en noir,
    il suffit de le relier au n\oe ud correspondant à la couleur 
    noire dans le graphe présenté ci-dessus.

    Nous allons commencer en choisissant trois couleurs :
    \begin{itemize}
      \item Vert : N\oe ud est vrai
      \item Rouge : No\oe ud est faux
      \item Noir : ne représente rien.
    \end{itemize}

    \columnbreak

    Nous représentons les variables et leur négation 
    par de simple N\oe uds. Nous relions chaque 
    variable à sa négation afin de garantir qu'elles 
    soient coloriées avec deux couleurs différentes.
    On relie ces n\oe uds au nœud noir afin d'empêcher 
    qu'ils soient coloriés en noir.

    "Une clause est représentée par un simple n\oe ud. 
    Pour la forcer à être verte, on la relie aux nœuds rouge et noir.

    Si une variable est coloriée en vert, elle est vraie. 
    Sinon, si sa négation est coloriée en vert, elle est fausse.
    \columnbreak


    Pour faire cette réduction on utilise ce ``gadget'' :

  \noindent\begin{minipage}{\linewidth}
    \centering
    \begin{tikzpicture}
      \node[circle, draw] (G1) at (0, 0) {};
      \node[circle, draw] (G2) at (1, 0) {};
      \node[circle, draw] (G3) at (2, 0.5) {};
      \node[circle, draw] (G4) at (1, 1) {};
      \node[circle, draw] (G5) at (0, 1) {};
      \draw (G1) -- (G2) -- (G3) -- (G4) -- (G5);
      \draw (G2) -- (G4);
    \end{tikzpicture}
    \end{minipage}
    Il va nous servir de porte logique ``ou''.

    On force le sommet à être rouge ou vert en le reliant au sommet noir.

    Si le sommet est vert alors il y forcément 
    un des ``pieds'' qui est vert aussi.

    On peut alors construire le graphe en utilisant 
    des ``gadgets'' pour chaque clause. Ces ``gadgets'' 
    permettent de relier tous les ``pieds'' des clauses aux 
    variables ou à leurs négations correspondantes.

    \noindent\begin{minipage}{\linewidth}
    \centering
    \begin{tikzpicture}
      \node at (-0.4, 0.5) {$\beta$};
      \node[circle, fill=red] (G1) at (0, 0) {};
      \node[circle, fill=green] (G2) at (1, 0) {};
      \node[circle, fill=red] (G3) at (2, 0.5) {};
      \node[circle, fill=black] (G4) at (1, 1) {};
      \node[circle, fill=green] (G5) at (0, 1) {};
      \draw (G1) -- (G2) -- (G3) -- (G4) -- (G5);
      \draw (G2) -- (G4);

      \node at (-0.4, 2) {$\alpha_3$};
      \node[circle, fill=red] (G1) at (0, 1.5) {};
      \node[circle, fill=black] (G2) at (1, 1.5) {};
      \node[circle, fill=red] (G3) at (2, 2) {};
      \node[circle, fill=green] (G4) at (1, 2.5) {};
      \node[circle, fill=red] (G5) at (0, 2.5) {};
      \draw (G1) -- (G2) -- (G3) -- (G4) -- (G5);
      \draw (G2) -- (G4);

      \node at (-0.4, 3.5) {$\alpha_2$};
      \node[circle, fill=red] (G1) at (0, 3) {};
      \node[circle, fill=black] (G2) at (1, 3) {};
      \node[circle, fill=green] (G3) at (2, 3.5) {};
      \node[circle, fill=red] (G4) at (1, 4) {};
      \node[circle, fill=green] (G5) at (0, 4) {};
      \draw (G1) -- (G2) -- (G3) -- (G4) -- (G5);
      \draw (G2) -- (G4);

      \node at (-0.4, 5) {$\alpha_1$};
      \node[circle, fill=green] (G1) at (0, 4.5) {};
      \node[circle, fill=black] (G2) at (1, 4.5) {};
      \node[circle, fill=green] (G3) at (2, 5) {};
      \node[circle, fill=red] (G4) at (1, 5.5) {};
      \node[circle, fill=green] (G5) at (0, 5.5) {};
      \draw (G1) -- (G2) -- (G3) -- (G4) -- (G5);
      \draw (G2) -- (G4);
    \end{tikzpicture}
    \end{minipage}

    Si le graphe est coloriable, alors les n\oe uds 
    des clauses doivent être coloriés en vert, ce qui 
    signifie que les gadgets doivent avoir au moins un 
    pied vert. Dans la branche verte, le coloriage utilisera 
    les configurations $\alpha_{1,2}$, car le sommet est forcément 
    vert. Par conséquent, la configuration $\beta$ ne pose pas de 
    problème pour la réduction.
  \end{multicols}
\end{document}
